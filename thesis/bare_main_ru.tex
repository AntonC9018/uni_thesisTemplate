\documentclass[a4paper,12pt]{report}

\usepackage[russian]{config}

% Description
\newcommand{\thesisTitleEng}{Название диссертации} % Doar pentru master
\newcommand{\uniGroupName}{MIA2201}

\newcommand{\authorName}{NUME PRENUME} % Fără patronimic
\newcommand{\thesisTitle}{Explorarea formatului PNG}
\newcommand{\thesisType}{master} % an / licență / master. <teză de> se adaugă automat.
\newcommand{\programulDeStudii}{master} % licență / master. <programul de> se adăugă automat.
\newcommand{\identificatorulCursului}{0613.5 Informatică aplicată} % 0211.7 Design jocurilor
\newcommand{\conducatorName}{NUME PRENUME}

\newcommand{\authorNameRu}{ФАМИЛИЯ ИМЯ ОТЧЕСТВО}
\newcommand{\thesisTitleRu}{Исследование формата PNG}
\newcommand{\thesisTypeRu}{магистерская} % бакалаврская / курсовая. <работа> вставится после сама
\newcommand{\programulDeStudiiRu}{магистра} % бакалавриата. <программа> вставится перед сама
\newcommand{\identificatorulCursuluiRu}{0613.5 Прикладная информатика} % 0211.7 Гейм дизайн
\newcommand{\conducatorNameRu}{ФАМИЛИЯ ИМЯ} % ИМЯ ФАМИЛИЯ

\newcommand{\conferencesList}{Студенческая конференция номер НОМЕР, \year~года}
\renewcommand{\year}{2024}
\newcommand{\github}{\url{https://github.com/USER/REPO}}
\newcommand{\outputDate}{\today}

\begin{document}

\titlePage

\clearpage
\tableofcontents

\clearpage
\unnumberedChapter{Аннотация} % Только для магистратуры

\textbf{к \thesisTypeRu{} диссертации ``\thesisTitleRu{}'', студента \authorNameRu{}, группа \uniGroupName{}, программа обучения \programulDeStudiiRu.}

\textbf{Структура диссертации.}
Диссертация состоит из: Введения, \chapterCount{} глав, Общих выводов и рекомендаций, Библиографии из \bibliographyEntryCount{} наименований.
Основной текст занимает \usefulPageCount{} страниц и включает \anexeCount{} приложения.

\textbf{Ключевые слова:}
\textit{\acs{PNG}, слово 1, слово 2}

\textbf{Актуальность.}

Где используются технологии, почему тема важна.

\textbf{Цель и задачи исследования.}

Цель = что вы пытаетесь доказать в этой диссертации.

Задачи = шаги, через которые вы это доказываете.

\textbf{Ожидаемые и полученные результаты} можно резюмировать в:
(1) изучение чего-то
(2) проектирование чего-то
(3) реализация чего-то.

\textbf{Важные решённые проблемы} включают: ...

\textbf{Практическая ценность.} В результате было получено ...

Весь исходный код проекта доступен на GitHub по следующей ссылке: \github.

Полученные результаты были представлены на \textbf{\conferencesList}\cite{self}.

\clearpage
\unnumberedChapter{Annotation} % Только для магистратуры

\textbf{of the \thesisType{} thesis ``\thesisTitleEng'' of the student \authorName{}, group \uniGroupName{}, \programulDeStudii{} study program.}

\textbf{The structure of the thesis.}

\textbf{Keywords:}
\textit{}

\textbf{Relevance.}

\textbf{Purpose and objectives of this research.}

\textbf{Important resolved problems} are:

\textbf{Applicative value.} 

The entire source code can be obtained by accessing the GitHub repository
by the following URL: \github. 

The obtained results were reported at \textbf{\conferencesList}\cite{self}.

\clearpage
\unnumberedChapter{Список сокращений}
\begin{acronym}[JPEG]
    % Добавьте сокращения здесь.
    % Добавляйте только упомянутые сокращения.
    % В дальнейшем всегда пишите \ac{PNG} вместо просто PNG.
    \acro{PNG}{Portable Network Graphics}
    \acro{PR}{Pull Request}
\end{acronym}

\introChapter

\textbf{Актуальность и важность темы.}

То же самое, что и в аннотации, но более подробно.

\textbf{Цель и задачи.}

То же самое.

\textbf{Методологическая и технологическая база.}

Какие библиотеки / инструменты / ресурсы вы использовали.

Какой подход вы выбрали.

Возможно, вдохновения.

\textbf{Научная новизна / оригинальность.}

Чем ваша диссертация / продукт отличается от других существующих приложений / решений / исследований.

\textbf{Практическая ценность.}

Для чего может быть использовано приложение.
То, что оно помогло вам изучить эти технологии, недостаточно,
нужно указать что-то помимо этого.

Можно также сделать такой список:
\begin{itemize}
    \item Ценность A
    \item Ценность B
    \item Ценность C
\end{itemize}

\textbf{Краткое содержание диссертации.}

Первая глава, \nameref{intro_chapter_title}, представляет общую / теоретическую информацию о ...

Вторая глава, \nameref{architecture_chapter_title}, описывает реализацию, ...

Третья глава, \nameref{implementation_chapter_title}, следует за реализацией ...

\chapter{Название теоретической главы}\label{intro_chapter_title}

\section{Синтаксис \LaTeX{}}

\subsection{Введение}

Здесь демонстрируется базовый синтаксис \LaTeX{}.

Очевидно, что подглавы такого размера неприемлемы,
нужно более подробно аргументировать вещи.

Например, можно написать, что базовый синтаксис важен
для написания даже примитивной работы на \LaTeX{}, и поскольку
цель — изучить \LaTeX{} на достаточно продвинутом уровне для написания
диссертации, эта информация необходима для перехода к более глубоким идеям.

\subsection{Цитирование источников}

Цитирование всего источника делается после упоминания, вот так\cite{gif_unusable_reason}.

Можно ссылаться на конкретные страницы или параграфы, вот так\cite[12.2.]{png_spec}.

\subsection{Синтаксис списков}

Ненумерованные. Используются для представления связанной информации:

\begin{itemize}
    \item
        Изображения с палитрой до 256 цветов.
        
    \item
        Возможность потоковой передачи:
        файлы могут читаться и записываться последовательно, что позволяет использовать формат
        PNG как протокол связи для динамической генерации и отображения изображений.

    \item
        Прогрессивное отображение: подготовленный файл изображения может быть
        отображён по мере получения по каналу связи,
        предоставляя очень быстро изображение с низким разрешением,
        с последующим постепенным улучшением деталей.

    \item
        Прозрачность: части изображения могут быть помечены как прозрачные,
        создавая эффект не прямоугольного изображения.

    \item 
        Дополнительная информация: текстовые комментарии и другие данные могут быть
        сохранены внутри файла изображения.

    \item
        Полная независимость от аппаратного обеспечения и платформы.

    \item
        Эффективное сжатие, 100\% без потерь.
\end{itemize}

Нумерованные. Используются, например, для списка последовательных шагов:

\begin{enumerate}
    \item 
        Изображения с палитрой до 256 цветов.
    \item 
        Возможность потоковой передачи:
        файлы могут читаться и записываться последовательно, что позволяет использовать формат
        PNG как протокол связи для динамической генерации и отображения изображений.
    \item 
        Прогрессивное отображение: подготовленный файл изображения может быть
        отображён по мере получения по каналу связи,
        предоставляя очень быстро изображение с низким разрешением,
        с последующим постепенным улучшением деталей.
    \item 
        Прозрачность: части изображения могут быть помечены как прозрачные,
        создавая эффект не прямоугольного изображения.
\end{enumerate}

\subsection{Стилизация}

\textbf{ЖИРНЫЙ ТЕКСТ = Жирный}

\textit{КУРСИВНЫЙ ТЕКСТ = Курсив}

\texttt{МОНОШИРНЫЙ ТЕКСТ = Моноширинный шрифт для кода}

Можно \textbf{применять} \textit{в предложении}, или \texttt{\textit{\textbf{комбинировать}}}.

\subsection{Таблицы}

Когда у вас есть таблица, необходимо дать на неё ссылку.
Пример\refFigure{pixel_order_table} таблицы:

\begin{figure}[!ht]
\centering
\begin{tabular}{c c c c c c c c}
    1 & 6 & 4 & 6 & 2 & 6 & 4 & 6 \\
    7 & 7 & 7 & 7 & 7 & 7 & 7 & 7 \\
    5 & 6 & 5 & 6 & 5 & 6 & 5 & 6 \\
    7 & 7 & 7 & 7 & 7 & 7 & 7 & 7 \\
    3 & 6 & 4 & 6 & 3 & 6 & 4 & 6 \\
    7 & 7 & 7 & 7 & 7 & 7 & 7 & 7 \\
    5 & 6 & 5 & 6 & 5 & 6 & 5 & 6 \\
    7 & 7 & 7 & 7 & 7 & 7 & 7 & 7 \\
\end{tabular}
\caption{Таблица порядка пикселей}
\label{fig:pixel_order_table}
\end{figure}

\section{Библиотека \texttt{minted} для кода}

\subsection{Общая информация}

Библиотека \texttt{minted} используется для стилизации кода.

\subsection{Прямая вставка блока кода}

\begin{minted}{cpp}
// любой код здесь
int main()
{
    std::cout << "hello";
}
\end{minted}

\subsection{Вставка целого файла}

\inputminted[]{zig}{../src/sourcefile.zig}

\subsection{Вставка части файла}

\inputminted[firstline=2,lastline=5]{zig}{../src/sourcefile.zig}

\subsection{Вставка сегмента из файла}

Эта функциональность реализована благодаря скрипту \texttt{findSegment.py}.
\texttt{minted} возможно добавит такой функционал в будущем (или уже добавил?).

Для настройки языка программирования файлов, изменяйте \texttt{findSegment.py}.

\inputMintedSegment{../src/sourcefile.zig}{example}

\section{Изображения}

\subsection{Введение}

Этот шаблон предлагает только одну функцию для изображений, 
которая также автоматически регистрирует его как фигуру.
Можно использовать и что-либо другое, если она вам не подходит.
Для стандартизации, однако, хорошо предложить свой метод через \ac{PR}, чтобы добавить его в шаблон.

\subsection{Центрированное изображение}

Когда вы добавляете изображение, делайте на него ссылку в тексте.

На \refFigure{interface_sketch.png} представлен эскиз графического интерфейса.

\imageWithCaption{interface_sketch.png}{Эскиз графического интерфейса}

\section{Приложения}

\subsection{Когда использовать приложение vs прямой код?}

Когда содержание кода не является строго необходимым
для его объяснения на высоком уровне в тексте.

Также, когда код слишком большой.
Попробуйте включать прямо в текст не более 50 строк.

\subsection{Как сделать ссылку}

См. приложение \ref{appendix:example_min}.

\chapterConclusionSection{intro_chapter_title}

Что обсуждалось в этой главе, кратко.

\chapter{Проектирование приложения}\label{architecture_chapter_title}

\section{Введение}

Здесь включите информацию о структуре приложения на высоком уровне,
возможно, некоторые диаграммы, почему вы использовали выбранные технологии и т.д.

\chapterConclusionSection{architecture_chapter_title}

\chapter{Реализация системы}\label{implementation_chapter_title}

Здесь представьте более подробную информацию о конкретных модулях реализованной системы.
Какие проблемы вы встретили и как их преодолели.

Как вы использовали важные фичи библиотек.

Продемонстрируйте ваше приложение, с изображениями — либо непосредственно в тексте,
если они важны для понимания и более специфичны,
либо поместите их в приложение, и вставьте ссылки в текст.

\section{Введение}

Здесь включите информацию о структуре приложения на высоком уровне,
возможно, некоторые диаграммы, почему вы использовали выбранные технологии и т.д.

\chapterConclusionSection{implementation_chapter_title}

В этой главе обсуждалось ...

\unnumberedChapter{Заключительные выводы и рекомендации}

Кратко повторите самое интересное.
Добавьте, что можно улучшить или где продукт может быть использован в будущем.

Пример завершённой диссертации:
\begin{itemize}
  \item Исходный код (программа, текст диссертации на \LaTeX{}): \url{https://github.com/AntonC9018/thesis-png}
  \item PDF: \url{https://drive.google.com/file/d/1ZiGQt6PvUm3FGY8oyIlVsERc4PvZHJu0/view?usp=drive_link}
\end{itemize}

Эти два пункта ниже включайте всегда:

Весь исходный код, включая программный код и текст работы
в форме до рендеринга, доступен на GitHub по следующей ссылке: \github.

Полученные результаты были представлены на \textbf{\conferencesList}\cite{self}.

\bibliographyChapter

\appendixChapter

\section{Функция min в тестовой программе на JavaScript}\label{appendix:example_min}
\inputminted{js}{../src/appendix_example.js}

\declarationPage{}

\end{document}
% vim: fdm=syntax
